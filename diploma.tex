%%%%%%%%%%%%%%%%%%%%%%%%%%%%%%%%%%%%%%%%
% datoteka diploma-FRI-vzorec.tex
%
%POZOR: ta verzija ne producira pdf datoteke v pdf/A formatu!!!
%namenjena je le za nalogo pri Diplomskem seminarju!
%
% vzorčna datoteka za pisanje diplomskega dela v formatu LaTeX
% na UL Fakulteti za računalništvo in informatiko
%
% na osnovi starejših verzij vkup spravil Franc Solina, maj 2021
% prvo verzijo je leta 2010 pripravil Gašper Fijavž
%
% za upravljanje z literaturo ta vezija uporablja BibLaTeX
%
% svetujemo uporabo Overleaf.com - na tej spletni implementaciji LaTeXa ta vzorec zagotovo pravilno deluje
%

\documentclass[a4paper,12pt,openright]{book}
%\documentclass[a4paper, 12pt, openright, draft]{book}  Nalogo preverite tudi z opcijo draft, ki pokaže, katere vrstice so predolge! Pozor, v draft opciji, se slike ne pokažejo!
 
\usepackage[utf8]{inputenc}   % omogoča uporabo slovenskih črk kodiranih v formatu UTF-8
\usepackage[slovene,english]{babel}    % naloži, med drugim, slovenske delilne vzorce
\usepackage[pdftex]{graphicx}  % omogoča vlaganje slik različnih formatov
\usepackage{fancyhdr}          % poskrbi, na primer, za glave strani
\usepackage{amssymb}           % dodatni matematični simboli
\usepackage{amsmath}           % eqref, npr.
\usepackage{hyperxmp}
\usepackage[hyphens]{url}
\usepackage{csquotes}
\usepackage[pdftex, colorlinks=true,
						citecolor=black, filecolor=black, 
						linkcolor=black, urlcolor=black,
						pdfproducer={LaTeX}, pdfcreator={LaTeX}]{hyperref}

\usepackage{color}
\usepackage{soul}

\usepackage[
backend=biber,
style=numeric,
sorting=nty,
]{biblatex}


\addbibresource{literatura.bib} %Imports bibliography file


%%%%%%%%%%%%%%%%%%%%%%%%%%%%%%%%%%%%%%%%
%	DIPLOMA INFO
%%%%%%%%%%%%%%%%%%%%%%%%%%%%%%%%%%%%%%%%
\newcommand{\ttitle}{Ponudniki identitev z uporabo pametnih pogodb}
\newcommand{\ttitleEn}{Identity providers backed by smart contract}
\newcommand{\tsubject}{\ttitle}
\newcommand{\tsubjectEn}{\ttitleEn}
\newcommand{\tauthor}{Rok Šerak}
\newcommand{\tkeywords}{Verige blokov, Pametne pogodbe, Ponudniki identitet, Ethereum, Filecoin, JavaScript, TypeScript, Solidity}
\newcommand{\tkeywordsEn}{Blockchain, Smart contracts, Identity providers, Ethereum, Filecoin, JavaScript, TypeScript, Solidity}

%%%%%%%%%%%%%%%%%%%%%%%%%%%%%%%%%%%%%%%%
%	HYPERREF SETUP
%%%%%%%%%%%%%%%%%%%%%%%%%%%%%%%%%%%%%%%%
\hypersetup{pdftitle={\ttitle}}
\hypersetup{pdfsubject=\ttitleEn}
\hypersetup{pdfauthor={\tauthor}}
\hypersetup{pdfkeywords=\tkeywordsEn}

%%%%%%%%%%%%%%%%%%%%%%%%%%%%%%%%%%%%%%%%
% postavitev strani
%%%%%%%%%%%%%%%%%%%%%%%%%%%%%%%%%%%%%%%%  

\addtolength{\marginparwidth}{-20pt} % robovi za tisk
\addtolength{\oddsidemargin}{40pt}
\addtolength{\evensidemargin}{-40pt}

\renewcommand{\baselinestretch}{1.3} % ustrezen razmik med vrsticami
\setlength{\headheight}{15pt}        % potreben prostor na vrhu
\renewcommand{\chaptermark}[1]%
{\markboth{\MakeUppercase{\thechapter.\ #1}}{}} \renewcommand{\sectionmark}[1]%
{\markright{\MakeUppercase{\thesection.\ #1}}} \renewcommand{\headrulewidth}{0.5pt} \renewcommand{\footrulewidth}{0pt}
\fancyhf{}
\fancyhead[LE,RO]{\sl \thepage} 
%\fancyhead[LO]{\sl \rightmark} \fancyhead[RE]{\sl \leftmark}
\fancyhead[RE]{\sc \tauthor}              % dodal Solina
\fancyhead[LO]{\sc Diplomska naloga}     % dodal Solina


\newcommand{\BibLaTeX}{{\sc Bib}\LaTeX}
\newcommand{\BibTeX}{{\sc Bib}\TeX}

%%%%%%%%%%%%%%%%%%%%%%%%%%%%%%%%%%%%%%%%
% naslovi
%%%%%%%%%%%%%%%%%%%%%%%%%%%%%%%%%%%%%%%%  

\newcommand{\autfont}{\Large}
\newcommand{\titfont}{\LARGE\bf}
\newcommand{\clearemptydoublepage}{\newpage{\pagestyle{empty}\cleardoublepage}}
\setcounter{tocdepth}{1}	      % globina kazala

%%%%%%%%%%%%%%%%%%%%%%%%%%%%%%%%%%%%%%%%
% konstrukti
%%%%%%%%%%%%%%%%%%%%%%%%%%%%%%%%%%%%%%%%  
\newtheorem{izrek}{Izrek}[chapter]
\newtheorem{trditev}{Trditev}[izrek]
\newenvironment{dokaz}{\emph{Dokaz.}\ }{\hspace{\fill}{$\Box$}}


%%%%%%%%%%%%%%%%%%%%%%%%%%%%%%%%%%%%%%%%%%%%%%%%%%%%%%%%%%%%%%%%%%%%%%%%%%%%%%%
%% PDF-A
%%%%%%%%%%%%%%%%%%%%%%%%%%%%%%%%%%%%%%%%%%%%%%%%%%%%%%%%%%%%%%%%%%%%%%%%%%%%%%%

%%%%%%%%%%%%%%%%%%%%%%%%%%%%%%%%%%%%%%%% 
% define medatata
%%%%%%%%%%%%%%%%%%%%%%%%%%%%%%%%%%%%%%%% 
\def\Title{\ttitle}
\def\Author{\tauthor, rok.serak@gmail.com}
\def\Subject{\ttitleEn}
\def\Keywords{\tkeywordsEn}

%%%%%%%%%%%%%%%%%%%%%%%%%%%%%%%%%%%%%%%% 
% \convertDate converts D:20080419103507+02'00' to 2008-04-19T10:35:07+02:00
%%%%%%%%%%%%%%%%%%%%%%%%%%%%%%%%%%%%%%%% 
\def\convertDate{%
    \getYear
}

{\catcode`\D=12
 \gdef\getYear D:#1#2#3#4{\edef\xYear{#1#2#3#4}\getMonth}
}
\def\getMonth#1#2{\edef\xMonth{#1#2}\getDay}
\def\getDay#1#2{\edef\xDay{#1#2}\getHour}
\def\getHour#1#2{\edef\xHour{#1#2}\getMin}
\def\getMin#1#2{\edef\xMin{#1#2}\getSec}
\def\getSec#1#2{\edef\xSec{#1#2}\getTZh}
\def\getTZh +#1#2{\edef\xTZh{#1#2}\getTZm}
\def\getTZm '#1#2'{%
    \edef\xTZm{#1#2}%
    \edef\convDate{\xYear-\xMonth-\xDay T\xHour:\xMin:\xSec+\xTZh:\xTZm}%
}

%\expandafter\convertDate\pdfcreationdate 

%%%%%%%%%%%%%%%%%%%%%%%%%%%%%%%%%%%%%%%%
% get pdftex version string
%%%%%%%%%%%%%%%%%%%%%%%%%%%%%%%%%%%%%%%% 
\newcount\countA
\countA=\pdftexversion
\advance \countA by -100
\def\pdftexVersionStr{pdfTeX-1.\the\countA.\pdftexrevision}


%%%%%%%%%%%%%%%%%%%%%%%%%%%%%%%%%%%%%%%%
% XMP data
%%%%%%%%%%%%%%%%%%%%%%%%%%%%%%%%%%%%%%%%  
\usepackage{xmpincl}
%\includexmp{pdfa-1b}

%%%%%%%%%%%%%%%%%%%%%%%%%%%%%%%%%%%%%%%%
% pdfInfo
%%%%%%%%%%%%%%%%%%%%%%%%%%%%%%%%%%%%%%%%  
\pdfinfo{%
    /Title    (\ttitle)
    /Author   (\tauthor, rok.serak@gmail.com)
    /Subject  (\ttitleEn)
    /Keywords (\tkeywordsEn)
    /ModDate  (\pdfcreationdate)
    /Trapped  /False
}

%%%%%%%%%%%%%%%%%%%%%%%%%%%%%%%%%%%%%%%%
% znaki za copyright stran
%%%%%%%%%%%%%%%%%%%%%%%%%%%%%%%%%%%%%%%%  

\newcommand{\CcImageCc}[1]{%
	\includegraphics[scale=#1]{cc_cc_30.pdf}%
}
\newcommand{\CcImageBy}[1]{%
	\includegraphics[scale=#1]{cc_by_30.pdf}%
}
\newcommand{\CcImageSa}[1]{%
	\includegraphics[scale=#1]{cc_sa_30.pdf}%
}

%%%%%%%%%%%%%%%%%%%%%%%%%%%%%%%%%%%%%%%%%%%%%%%%%%%%%%%%%%%%%%%%%%%%%%%%%%%%%%%
%%%%%%%%%%%%%%%%%%%%%%%%%%%%%%%%%%%%%%%%%%%%%%%%%%%%%%%%%%%%%%%%%%%%%%%%%%%%%%%

\begin{document}
\selectlanguage{slovene}
\frontmatter
\setcounter{page}{1} %
\renewcommand{\thepage}{}       % preprečimo težave s številkami strani v kazalu

%%%%%%%%%%%%%%%%%%%%%%%%%%%%%%%%%%%%%%%%
%naslovnica
 \thispagestyle{empty}%
   \begin{center}
    {\large\sc Univerza v Ljubljani\\%
%      Fakulteta za elektrotehniko\\% za študijski program Multimedija
%      Fakulteta za upravo\\% za študijski program Upravna informatika
      Fakulteta za računalništvo in informatiko\\%
%      Fakulteta za matematiko in fiziko\\% za študijski program Računalništvo in matematika
     }
    \vskip 10em%
    {\autfont \tauthor\par}%
    {\titfont \ttitle \par}%
    {\vskip 3em \textsc{DIPLOMSKO DELO\\[5mm]         % dodal Solina za ostale študijske programe
%    VISOKOŠOLSKI STROKOVNI ŠTUDIJSKI PROGRAM\\ PRVE STOPNJE\\ RAČUNALNIŠTVO IN INFORMATIKA}\par}%
     VISOKOŠOLSKI  ŠTUDIJSKI PROGRAM\\ PRVE STOPNJE\\ RAČUNALNIŠTVO IN INFORMATIKA}\par}%
%    INTERDISCIPLINARNI UNIVERZITETNI\\ ŠTUDIJSKI PROGRAM PRVE STOPNJE\\ MULTIMEDIJA}\par}%
%    INTERDISCIPLINARNI UNIVERZITETNI\\ ŠTUDIJSKI PROGRAM PRVE STOPNJE\\ UPRAVNA INFORMATIKA}\par}%
%    INTERDISCIPLINARNI UNIVERZITETNI\\ ŠTUDIJSKI PROGRAM PRVE STOPNJE\\ RAČUNALNIŠTVO IN MATEMATIKA}\par}%
    \vfill\null%
% izberite pravi habilitacijski naziv mentorja!
    {\large \textsc{Mentor}: Doc. Dr. Luka Šajn \par}%
    {\vskip 2em \large Ljubljana, \the\year \par}%
\end{center}
% prazna stran
%\clearemptydoublepage      
% izjava o licencah itd. se izpiše na hrbtni strani naslovnice

%%%%%%%%%%%%%%%%%%%%%%%%%%%%%%%%%%%%%%%%
%copyright stran
%%%%%%%%%%%%%%%%%%%%%%%%%%%%%%%%%%%%%%%%
\newpage
\thispagestyle{empty}

\vspace*{5cm}
{\small \noindent
To delo je ponujeno pod licenco \textit{Creative Commons Priznanje avtorstva-Deljenje pod enakimi pogoji 2.5 Slovenija} (ali novej\v so razli\v cico).
To pomeni, da se tako besedilo, slike, grafi in druge sestavine dela kot tudi rezultati diplomskega dela lahko prosto distribuirajo,
reproducirajo, uporabljajo, priobčujejo javnosti in predelujejo, pod pogojem, da se jasno in vidno navede avtorja in naslov tega
dela in da se v primeru spremembe, preoblikovanja ali uporabe tega dela v svojem delu, lahko distribuira predelava le pod
licenco, ki je enaka tej.
Podrobnosti licence so dostopne na spletni strani \href{http://creativecommons.si}{creativecommons.si} ali na Inštitutu za
intelektualno lastnino, Streliška 1, 1000 Ljubljana.

\vspace*{1cm}
\begin{center}% 0.66 / 0.89 = 0.741573033707865
\CcImageCc{0.741573033707865}\hspace*{1ex}\CcImageBy{1}\hspace*{1ex}\CcImageSa{1}%
\end{center}
}

\vspace*{1cm}
{\small \noindent
Izvorna koda diplomskega dela, njeni rezultati in v ta namen razvita programska oprema je ponujena pod licenco GNU General Public License,
različica 3 (ali novejša). To pomeni, da se lahko prosto distribuira in/ali predeluje pod njenimi pogoji.
Podrobnosti licence so dostopne na spletni strani \url{http://www.gnu.org/licenses/}.
}

\vfill
\begin{center} 
\ \\ \vfill
{\em
Besedilo je oblikovano z urejevalnikom besedil \LaTeX.}
\end{center}

% prazna stran
\clearemptydoublepage

%%%%%%%%%%%%%%%%%%%%%%%%%%%%%%%%%%%%%%%%
% stran 3 med uvodnimi listi
\thispagestyle{empty}
\
\vfill

\bigskip
\noindent\textbf{Kandidat:} Rok Šerak \\
\noindent\textbf{Naslov:} Ponudniki identitev z uporabo pametnih pogodb \\
% vstavite ustrezen naziv študijskega programa!
\noindent\textbf{Vrsta naloge:} Diplomska naloga na visokošolskem programu prve stopnje Računalništvo in informatika \\
% izberite pravi habilitacijski naziv mentorja!
\noindent\textbf{Mentor:} Doc. Dr. Luka Šajn \\

\bigskip
\noindent\textbf{Opis:}\\
Besedilo teme diplomskega dela študent prepiše iz študijskega informacijskega sistema, kamor ga je vnesel mentor. 
V nekaj stavkih bo opisal, kaj pričakuje od kandidatovega diplomskega dela. 
Kaj so cilji, kakšne metode naj uporabi, morda bo zapisal tudi ključno literaturo.

\bigskip
\noindent\textbf{Title:} Identity providers backed by smart contract

\bigskip
\noindent\textbf{Description:}\\
opis diplome v angleščini

\vfill



\vspace{2cm}

% prazna stran
\clearemptydoublepage

% zahvala
\thispagestyle{empty}\mbox{}\vfill\null\it%
\noindent
Na tem mestu zapišite, komu se zahvaljujete za pomoč pri izdelavi diplomske naloge oziroma pri vašem študiju nasploh. Pazite, da ne boste koga pozabili. Utegnil vam bo zameriti. Temu se da izogniti tako, da celotno zahvalo izpustite.
\rm\normalfont

% prazna stran
\clearemptydoublepage

%%%%%%%%%%%%%%%%%%%%%%%%%%%%%%%%%%%%%%%%
% posvetilo, če sama zahvala ne zadošča :-)
\thispagestyle{empty}\mbox{}{\vskip0.20\textheight}\mbox{}\hfill\begin{minipage}{0.55\textwidth}%
Svoji dragi Alenčici.
\normalfont\end{minipage}

% prazna stran
\clearemptydoublepage


%%%%%%%%%%%%%%%%%%%%%%%%%%%%%%%%%%%%%%%%
% kazalo
\pagestyle{empty}
\def\thepage{}% preprečimo težave s številkami strani v kazalu
\tableofcontents{}


% prazna stran
\clearemptydoublepage

%%%%%%%%%%%%%%%%%%%%%%%%%%%%%%%%%%%%%%%%
% seznam kratic

\chapter*{Seznam uporabljenih kratic}

\noindent\begin{tabular}{p{0.11\textwidth}|p{.39\textwidth}|p{.39\textwidth}}    % po potrebi razširi prvo kolono tabele na račun drugih dveh!
  {\bf kratica} & {\bf angleško}                              & {\bf slovensko} \\ \hline
  {\bf CA}      & classification accuracy               & klasifikacijska točnost \\
  {\bf DBMS} & database management system & sistem za upravljanje podatkovnih baz \\
  {\bf SVM}   & support vector machine              & metoda podpornih vektorjev \\
%  \dots & \dots & \dots \\
\end{tabular}


% prazna stran
\clearemptydoublepage

%%%%%%%%%%%%%%%%%%%%%%%%%%%%%%%%%%%%%%%%
% povzetek
\addcontentsline{toc}{chapter}{Povzetek}
\chapter*{Povzetek}

\noindent\textbf{Naslov:} \ttitle
\bigskip

\noindent\textbf{Avtor:} \tauthor
\bigskip

%\noindent\textbf{Povzetek:} 
\noindent V vzorcu je predstavljen postopek priprave diplomskega dela z uporabo okolja \LaTeX. Vaš povzetek mora sicer vsebovati približno 100 besed, ta tukaj je odločno prekratek.
Dober povzetek vključuje: (1) kratek opis obravnavanega problema, (2) kratek opis vašega pristopa za reševanje tega problema in (3) (najbolj uspešen) rezultat ali prispevek diplomske naloge.

\bigskip

\noindent\textbf{Ključne besede:} \tkeywords.
% prazna stran
\clearemptydoublepage

%%%%%%%%%%%%%%%%%%%%%%%%%%%%%%%%%%%%%%%%
% abstract
\selectlanguage{english}
\addcontentsline{toc}{chapter}{Abstract}
\chapter*{Abstract}

\noindent\textbf{Title:} \ttitleEn
\bigskip

\noindent\textbf{Author:} \tauthor
\bigskip

%\noindent\textbf{Abstract:} 
\noindent This sample document presents an approach to typesetting your BSc thesis using \LaTeX. 
A proper abstract should contain around 100 words which makes this one way too short.
\bigskip

\noindent\textbf{Keywords:} \tkeywordsEn.
\selectlanguage{slovene}
% prazna stran
\clearemptydoublepage

%%%%%%%%%%%%%%%%%%%%%%%%%%%%%%%%%%%%%%%%
\mainmatter
\setcounter{page}{1}
\pagestyle{fancy}

\chapter{Uvod}

\section{Prestavitev problema}
Verige blokov same po sebi ponuja določeno mero anominmnosti uporabnikom.
Vsak uporabnik ob kreaciji računa na verigi blokov dobi zasebni ključ (ang. Private key), s katerim lahko dokazuje lastništvo nad svojim računom.
Zasebni ključ uporabniki ohranjajo skrit, saj imaš z njim popoln nadzor na določenim računom.
Iz zasebnega ključa pa lahko izpeljemo javni ključ oziroma naslov, katerega ponavadi delimo z drugimi uporabniki, 
da nam lahko na primer pošljejo neko vsoto kripto valut na naš račun.
Tudi javni ključ načeloma ni javno povezan z neko osebo, razen če oseba to sama objavi oziroma izpostavi.
Tu vstopimo mi, z našo storitvijo ponudnika identitete, katero bomo implementirali z pametnimi pogodbami.
Uporabnik bo lahko v naši storitvi shranil na verigi blokov svoje osebne podatke, katere bo sam izbral, 
ter si tako ustvaril javno identito za svoj račun oziroma naslov na verigi blokov.

\section{Cilj diplome}
V sklopu diplome bomo implementirali storitev ponudnika identitev (ang. Identity provider) z uporabo pametnih pogodb (ang. Smart contract).
Storitev bo uporabnikom omogočala shranjenje, prikazovanje in urejanja svoje identitete, katero bodo lahko decentralizirane aplikacije za prikaz osnovnih podatkov o uporabniku.
Uporabnik sam bo seveda imel popoln nadzor nad svojo identiteto in tem, kaj želi, da njegova identiteta vključuje.
Torej noben vnos podatkov ne bo obvezen, uporabnik bo lahko prosto uporabljal našo storitev, v takšnem obsegu, kot bo to sam želel.
Storitev bo ponujala samo identiteto, ne pa tudi authentikacije in verifikacije, kar omogoča večina preostalih ponudnikov identitet, to pa zato, 
ker verige blokov že same po sebi ponujajo authentikacijo z uporabo privatnega ključa in pa tudi identito z javnim ključem, 
a iz javnega ključa težko razberemo kakršne koli osebne podatke o samem uporabniku, kar pa bo omogočala naša storitev.
Našo storitev bomo razvili na Ethereum verigi blokov in Filecoin decentraliziranem omrežju za shranjevanje datotek.
Na Ethereum platformi bomo imeli nameščeno našo storitev, ter hranili tekstovne podatke o uporabikih.
Na Filecoin omrežju pa bomo hranili uporabnikove datoteke, katere so prav tako del njegove identitete, kot je na primer profilna slika.
Dodatno bomo implementirali še spletni gradnik z uporabo React knjižnice, katerega bomo lahko decentralizirane aplikacije enostavno vključile vase.
Z gradnikom bodo lahko prikazali uporabnikove podatke, prav tako pa bo uporabnik lahko urejal podatke preko gradnika.
Seveda pa bo komunikacija z našo storitvijo možna tudi preko kateremo koli knjižnice, ki ima podporo za komunikacijo z pametnimi pogodbami na Ethereum verigi blokov, kot so recimo Ethers.js ali pa Web3.js.

\section{Pregled sorodnih del}
Google, Facebook, Apple sign in, OpenID, Keycloak?
TODO
Namecoin


\chapter{Orodja in tehnologije}
TODO

\section{Ponudniki identitet}
TODO

\section{Decentralizirane aplikacije}
Decentralizirane aplikacije so aplikacije, ki delujejo znotraj decenraliziranih platform.
Trenutno so najbolj popularne decentralizirane aplikacije na Ethereum verigi blokov.
Zaradi decentralizirane narave te aplikacije ne morejo biti manipulirane, odstranjene ali pa cenzurirane z strani 
nekega posameznika ali organizacije oziroma je to v praksi skoraj nemogoče izvesti.
Edin način za zaustavitev decentalizirane aplikacije, ki deluje na verigi blokov, bi bilo, da prevzamejo nadzor na verigo blokov z 51 \% napadom, 
kjer bi lahko posameznik zaradi večinskega nadzora na verigi blokov, preuredil že potrjene bloke, oziroma transakcije znotraj blokov, ki so 
komunicirale z določeno decentralizirano aplikacijo. 
V teoriji pa bi bilo tudi možno ustaviti neko decentalizirano aplikacijo, če bi v samih pametnih pogodbah, ki so del aplikacije, implementirali
neko funkcionalnost, ki bi ustavila, onemogočila delovanje samih pogodb.
Do te funkcionalnosti, bi verjetno imelo dostop le omejeno število računov, verjetno le lastnih same aplikacije, 
v tem primeru bi lahko sam lastnik ustavil svojo aplikacijo.
Seveda pa bi bilo možno odkriti ali takšna funkcionalnost obstaja znotraj določene aplikacije, 
saj imamo na verigah blokov možnost da posamezne pametne pogodbe dekompiliramo, ter tako analiziramo samo delovanje pogodb.
Ker delujejo na verigi blokov, ima do nje dostop vsak, ki ima račun na tisti verigi blokov, kjer ta aplikacija teče.
Videli smo že več primerov teh aplikacij na verigah blokov, kot so recimo decentralizirane menjalnice valut (DEX),
decentralizirane avtonomne organiziracije (DAO), začetna ponudba kovancev (ICO).
Je pa potrebno poudariti, da so decentralizirane aplikacije načeloma dražje za manjše uporabnike, kot pa tradicionalne centralizirane aplikacije, 
vsaj na Ethereum platformi, saj interakcija z verigo blokov, bolj specifično zapisovanje podatkov na verigo blokov ni poceni operacija.
Pri centraliziranih aplikacijah pa dosti stroškov pokrije sama centralizirana organizacija, ki stoji za aplikacijo \cite{dapps_investopedia} \cite{51_attack_investopedia}.


\section{Verige blokov}
Verige blokov so porazdeljene podatkovne baze, kjer so podatki porazdeljeni med vozlišča v njenem omrežju.
Verige blokov so trenutno najbolj razširjene v sistemih kriptovalut, kot so na primer Bitcoin in Ethereum.
Njihova vloga v tem sistemih je decentralizirano beleženje in potrjevanje transakcij.
Verige blokov zagotavljajo varnost in intergriteto podatkov brez potrebe po neki tretji entiteti, kakor je drugače zagotavljala varnost in intergriteto podatkov v centraliziranih podatkovnih bazah.
V verigah blokov so podatki razdeljeni v tako imenovane bloke.
Vsak blok drži določen del informacij oziroma informacije o stanju verige blokov v nekem trenutku.
Vsak blok je vezan na vse prejšnje bloke, torej ko se doda nov blok v verigo blokov, ga ni več možno spremeniti, ne vsebine, kot tudi ne položaja v verigi oziroma zaporedne številke bloka.
Ko želimo brati podatke na verigi blokov, moramo vedeti za kateri trenutek želimo dobiti podatke, lahko beremo podatke oziroma stanje v zadnjem, aktualnem bloku, lahko pa tudi beremo zgodovinske podatke, kakšno je bilo stanje 1 dan, 1 teden, 1 leto nazaj, vse do začetka verige do prvega bloka.
V tem primeru moramo samo vedeti katera številka bloka je bilo v tistem trenutku aktualno stanje \cite{blockchain_explained_investopedia}.

\section{Ethereum}
Ethereum je odprto kodna decentralizirana veriga blokov.
V času pisanja je Ethereum druga največja veriga blokov, večja je le Bitcoin veriga blokov, katera je tudi originalna oziroma prva implementacija verige blokov.
Kot primarno kripto valutu Ethereum uporablja Ether kriptovaluto.
Ethereum platforma vsebuje Ethereum Virtual Machine (EVM), katera omogoča izvajanje pametnih pogodb na sami verigi blokov.
Primarni programski jezik za uporabo znotraj EVM je Solidity, ostaja pa tudi Vyper programski jezik, ki pa je v času pisanja še v Alpha fazi razvoja.
Interakcija z Ethereum platformo in pametnimi pogodba na sami platformi pa je možna z uporabo katerega koli programskega jezika, potrebujemo le implementacijo komunikacije z samo platformo preko JSON RPC protokola in povezavo z katerim koli vozliščem, ki je del Ethereum platforme.
Obstajajo več implementacij Ethereum vozlišč, med katerimi so trenutno najpogosteje uporabljene Geth ali pa Nethermind implementacije.
Za generacijo novih blokov Ethereum uporablja Proof of work kriptografski dokaz Ethash, z katerim rudarji potrjujejo transakcije, ki se dogajajo znotraj Ethereum platforme. 
Ethereum transakcije lahko vključujejo različne informacije. 
Za potrjevanje trasakcij oziroma generacijo blokov, so rudarji tudi nagrajeni z izplačili v Ether kriptovaluti.
Lahko so le transkcije, ki pošljejo določeno količino Ether kriptovalute iz nekega računa na nek drug račun.
Lahko pa so tudi bolj kompleksne, te navadno vključujejo interakcijo z EVM in pametnimi pogodbami.
Je pa tudi v transakcijah, ki komunicijajo z EVM navadno vključena neka količina Ethera, saj je potrebno za spremembe stanja na verigi blokov plačati \cite{ethereum_whitepaper}.

TODO eth literatura

\section{Filecoin}
Filecoin je decentralizirano omrežje za hranjenje in distribucijo datotek.
Omrežje deluje na svoji verigi blokov z kriptovaluto Filecoin.
Prav tako omrežje deluje s pomočjo IPFS protokola \cite{IPFS}, ki omogoča učinkovito distribucijo datotekek na decentraliziranih omrežjih.
Na Filecoin omrežju lahko najamemo prostor za hranjenje datotek, za katerega plačamo s Filecoin kriptovaluto.
V omrežje se lahko priključi kdor koli, mora le ponuditi neko količino prostora na svojem sistemu, katero lahko Filecoin omrežje potem zasede za svojo rabo.
Filecoin uporablja tako imenovani Proof of replication dokaz, ki tu nadomešča Proof of work dokaz, katerega recimo uporablja Ethereum.
Proof of replication dokaz zagotavlja uporabnikom, da lahko dostopajo do svojih datotek na omrežju od kjer koli želijo.
Ta dokaz zagotavljajo tako imenovani rudarji, ki namesto potrjevanja transakcij, kar je njihova naloga pri verigah blokov kot je Ethereum, tukaj
zagotavljajo hranjenje in dostopnost do shranjenih datotek.
Za to so seveda nagrajeni z plačili oziroma nagradami, katere se rudarjem izplačajo v Filecoin kriptovaluti.
Poskrbljeno je tudi za varnost datotek, torej vse datoteke na omrežju so zašifrirane z ključem, do katerega ima dostop le uporabnik sam, kar pomeni, 
da ima dostop do same vsebine datotek le tisti, ki je te datoteke tudi naložil na omrežje \cite{filecoin}.

\section{Pametne pogodbe}
Pametne pogodbe so programi, kateri tečejo na verigah blokov.
Glavna razlika med pametnimi pogodbami, ter tradicionalnimi programi je to, da je pametna pogodba shranjena na verigi blokov, med tem ko so tradicionalni programi lahko shranjeni oziroma delujejo na kateri koli napravi.
Prav tako so vsi podatki, katere pametna pogodba potrebuje za svoje delovanje, shranjeni na verigi blokov.
Delovanje pametnih pogodb je avtomatizirano, kar pomeni, da ko nekdo želi izvesti neko funkcionalnost pametne pogodbe, ter zagotovi pogoje za izvedbo izbrane funkcionalnosti, te izvedbe ne more nihče preprečiti.
Za interakcijo z pametno pogodbo uporabnik potrebuje račun na verigi blokov, na kateri je pogodba nameščena.
Interakcijo oziroma akcije pametnih pogodb lahko razdelimo na dve večji kategoriji in sicer akcije, za katere je potrebno plačilo in tiste interakcije, katere so brezplačne.
Plačljive akcije so tiste, katerih izvedba ima vpliv na stanje verige blokov, torej bo spremenila oziroma posobodila trenutno stanje verige blokov.
Pomembno se je tudi zavedati, da se plačljive akcije ne izvede instantno, ampak je za izvedbo potrebna vključitev akcije v blok, katerega potem veriga blokov sprejme, kot novo aktualno stanje verige blokov.
Trenutno najbolj pogost primer uporabe pametnih pogodb so tako imenovani ERC20 tokni, katere lahko naredi kdor koli, ter ji nato znotraj pametne pogodbe razdeli ostalim uporabnikom. Pri ERC20 token pogodba je recimo plačljiva akcija prenov neke količine toknov iz računa enega uporabnika na račun drugega uporabnika.
Brezplačne akcije pa so tiste katerih izvedba le bere trenutno stanje verige blokov, ne naredi nobenih sprememb.
Prav tako pri brezplačnih akcijah ni potrebe po čakanju na potrditev akcije v naslednjem bloku, saj le beremo aktualno stanje in beremo lahko stanje kadarkoli in to instantno.
Primer brezplačne akcije pri ERC20 token pogodbah, katere smo omenili pri plačljivih akcijah, je recimo branje stanja oziroma količine toknov, katere si nek uporabnik trenutno lasti \cite{eth_smart_contract_intro}, \cite{erc20_token_standard}.

\section{Solidity}
Solidity je programski jezik narejen specifično za implementacijo pametnih pogodb znotraj Ethereum Virtual Machine.
Je objektno orientiran, statično tipiziran in podpira vse funkcionalnosti, katere se pričakuje danes od programskih jezikov,
kot so na primer dedovanje, knjižnice ali pa možnost definiranja kompleksnih tipov spremenljivk.
Prvotno je bil razvit za namene razvoja pametnih pogodb na Ethereum verigi blokov, a je sedaj še mnogo bolj razširjen,
saj ga lahko uporabimo za razvoj pametnih pogodb na kateri koli platformi, ki ima integriran Ethereum Virtual Machine \cite{solidity_docs}.

\section{JavaScript}
Javascript je objektno orientiran skriptni programski jezik, ki je bil prvotno razvit za izvajanje v spletnih brskalnikih.
Kasneje pa se ga je prilagodilo še za druga okolja, kot so recimo spletni strežniki z Node.js okoljem, znotraj PDF
dokumentov z uporabo Adobe Acrobat programske opreme in celo nekaterih podatkovnih bazah, kot je recimo Apache CouchDB.
Je pa še vedno najbolj pogost primer uporabe JavaScript programskega jezika znotraj spletnih aplikacij.
V spletnih aplikacijah nam ponuja zelo veliko funkcionalnosti, kot so recimo zagotavljanje interaktivnosti spletnih
aplikacij, dinamično prikazovanje podatkov in še veliko več.
Vse to nam omogočajo različni spletni APIjev, katere nam ponujajo sodobni spletnih brskalniki.
Prav preprost dostop in uporaba teh APIjev je zelo enostavna preko JavaScripta, kar je tudi glavni razlog za njegovo
trenutno visoko priljubljenostjo med podjetji in programerji \cite{javascript_mozilla}.

\section{TypeScript}
Typescript je močno tipiziran programski jezik narejen na osnovi Javascripta. 
Omogoča izdelavo aplikacij za vsa okolja, kjer lahko poganjamo Javascript aplikacije.
Typescript programe je potrebno prevesti, preden lahko program poženemo.
Preveden program je Javascript program v katerem so spremenljivke in funkcije bile preverjen za pravilno tipizacijo 
pred prevajanjem, kar zmanjša možnost napak zaradi uporabe spremenljivk oziroma podatkov napačnega tipa v programu, 
kar je pogost vzrok za napake pri programih izdelanih z pomočjo dimaničnih programskih jezikov, kot je Javascript \cite{typescript_homepage}.

\section{React}
React je JavaScript knjižnica namenjena izdelavi uporabniških vmesnikov za spletne aplikacije z uporabo spletnih tehnologij, kot so JavaScript, HTML, CSS.
Uporabimo ga lahko za izdelavo celotnih spletnih strani ali pa le za posamezne gradnike, katere se lahko vključi v 
druge spletne aplikacije, ki so lahko prav tako izdelane z React knjižnico ali pa katero koli drugo tehnologijo.
Prav tako nam omogoča preprosto razdelitev aplikacij v manjše komponente, katere je potem lažje sprogramirati, sestaviti skupaj v uporabniški vmesnik
ter morda še najbol pomembno, lažje vzdrževanje, kot kompleksnost aplikacije začne naraščati.
Mi ga bomo uporabili pri izdelavi gradnika za decentralizirane aplikacije z prikazom podatkov o uporabniku in obrazcem za urejanje njegovih podatkov \cite{react_homepage}


\section{WebStorm}
WebStorm je integrirano razvojno okolje podjetja JetBrains.
Namenjeno je razvoju spletnih aplikacij. 
Ponuja veliko različnih funkcionalnosti, kot so recimo sintaktično barvanje programske kode, sintaktično in semantično
analizo kode, razhroščevanje programske kode, sofisticirano refaktoriranje programske kode in še veliko več.
Ponuja dobro podporo za razvoj spletnih aplikacij, tako prednega dela (ang. Frontend) z ogrodji, kot so recimo React, Vue.js ali pa Angular
in zalednega dela (ang. Backend) z ogrodji, kot je recimo Express.js.
Kar se tiče programskih jezikov, pa ponuja podporo za JavaScript, TypeScript, Node.js in še nekaj drugih \cite{jetbrains_webstorm}.


\chapter{Razvoj aplikacije}
TODO


\chapter{Rezultati testiranja}
TODO


\chapter{Sklep}
TODO



%\cleardoublepage
%\addcontentsline{toc}{chapter}{Literatura}

\printbibliography[heading=bibintoc,type=article,title={Članki v revijah}]

\printbibliography[heading=bibintoc,type=inproceedings,title={Članki v zbornikih}]

\printbibliography[heading=bibintoc,type=incollection,title={Poglavja v knjigah}]

\printbibliography[heading=bibintoc,title={Celotna literatura}]


\end{document}
